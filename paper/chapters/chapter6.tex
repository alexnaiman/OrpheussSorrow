
This paper tries to solve a problem that the aspiring musician has always been fighting,
the music equivalent of "writer's block".
It provides a tool whose primary purpose is to enhance its creational
workflow by taking one of the most challenging tasks off his hands.
Orpheus's Sorrow application offers an easy way of composing songs based on a given feeling,
without the need for mastering music theory.

This trial represents only a proof of concept;
this paper's work is merely scratching the surface of what musical composition implies,
due to solely the complexity of the process,
but thanks to the results,
it has proved itself worthy of investing more time.

The model is capable of composing new songs with a well-defined structure,
based on the input emotion data and features.
Unfortunately, due to the small dataset the network was trained on,
the actual emotions of the generated songs are not that clear.
Despite that,
general characteristics of a song composed based on the given emotions were
found on the newly created compositions,
meaning that the model was able to capture some information about the emotion data.

For future developments,
one of the main improvements for this approach will be to enhance the dataset
with more songs with normal-distributed emotion data;
in this way, the prediction will be more accurate.
Hence the songs will reflect the emotions better.

Another future improvement can be the ability to compose starting
from an already written song,
being able to enhance or continue that piece of music.
This feature can be developed by using the encoder part of the model to predict the
encoding of the given song, attach the emotion data or tweak the values of the encoding,
and then plug the result in the decoder generating a new song.

The client app can also be improved with other features regarding the MIDI player.
One of those could be the ability to edit the created song,
change the playback instrument, and export the new piece as a MIDI file.
