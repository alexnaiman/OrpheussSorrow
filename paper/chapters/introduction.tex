“Music is a moral law. It gives a soul to the Universe, wings to the mind, flight to the imagination, a charm to sadness, gaiety and life to everything.
It is the essence of order, and leads to all that is good and just and beautiful.” \cite{quotationDictionary}

\section{Motivation}

The art of composing music is not one that any beginner can master
due to the fact that music theory has a very steep learning curve.

One can still compose music without knowing anything about theory,
but it will be tough to know how to really express their thoughts
since the notes do not have any meaning to them other than their sound.
When they start learning, they will find out about scales, modes,
chords progression, and how they clearly relate to some feelings.


\section{Solution Overview}

In the latest years, with the technological advancements regarding artificial intelligence,
dreaming about diagnosing a patient in seconds, facial recognition, instantaneous translating,
and self-driving cars is no longer an idea in Asimov’s mind.
Nowadays, with a bit of work, one can create a neural network,
train it on a given dataset and see the results about its objective,
without needing to worry about the internal structure of each component inside the network.

% \vspace{0.5cm}
This paper wants to encourage and help all people that
have difficulties composing music with the help of machine learning.
This paper aspires to create an application that can compose, enhance,
or continue a piece of music based on some emotions that the user can provide.
The music should have a structure, and the app will generate a
song with the given characteristics.

\section{Thesis Outline}
In this section, it will be discussed a sketch of the main chapters.

\textbf{The second chapter} centers around the theoretical aspects of machine learning
and deep learning.
It begins with an introduction to different machine learning styles,
how an artificial neural network works and its structure,
ending with a small insight into deep learning,
and the primary type of network this paper's work is about, the autoencoder.

In \textbf{chapter 3},
the paper focuses on related work and on other articles or projects
that have similar goals or approaches. There are treated three strategies,
each for different reasons. The first one, LahkNess \cite{donahue2019lakhnes},
is presented as state-of-the-art for multi-instrumental musical composition.
The second one represents one of the inspirations for this approach, the Composer \cite{hackerPoet},
which uses autoencoders. The last one, AIVA \cite{aiva}, has a similar goal,
to compose songs with a well-defined structure,
the difference being that it is based on genre.

\textbf{Chapter 4} contains the proposed approach.
It is a detailed view of the main structure of the model,
dataset it was trained on, and the results, meaning loss, accuracy,
and the way it composes.

\textbf{The fifth chapter} discussed the technical
implementation of the paper's application,
a description of the main frameworks used,
and the patterns applied.
There are also presented the main use cases along with their sequence diagrams,
and the user interface.

\textbf{The last chapter} describes the conclusions of this paper.
It presents a summary of the model's
results and also suggestions for future developments.
